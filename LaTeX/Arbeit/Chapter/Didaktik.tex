\chapter{TLS in der Lehre}

\section{Warum eignet sich TLS bzw. warum sollte man sich überhaupt damit befassen?}

\begin{mdframed}
\begin{itemize}
	\item "'einfaches"' Protokoll, das gut zur Erklärung einzelner Verfahren dienen kann (Record-Protocol für vertrauliche und authentifizierte Nachrichtenübertragung, Handshake zur Schlüsselaushandlung, ...)
	\item Viele Angriffe, die Abwehrmaßnahmen und auch "'Prinzip des schwächsten Kettenglieds"' deutlich machen (CBC-IV, Padding, MAC-then-Encrypt)
	\item weite Verbreitung -> notwendiges Verständnis
\end{itemize}
\end{mdframed}

\section{Fundamentale Ideen / Schwerpunkte für die Lehre}

\begin{mdframed}

Was ist das? -> Im kleineren Rahmen für TLS (wahrscheinlich nicht ganz passend, aber es geht um die Richtung)

Was könnte hier erwähnt werden?
\begin{itemize}
	\item Hybride Kryptosysteme (Symmetrisch und asymmetrisch)
	\item Problem des Schlüsselaustausches und Authentifizierung
	\item Familien von Angriffen betrachten?\\
	- Timing/Seitenkanal-Angriffe: Bleichenbacher und weitere\\
	- Replay-Angriffe: Sequenznummer
	\item Krypto-Standardvorgehen und Probleme, wenn man dagegen verstößt: Zufällige IVs in CBC, MAC-then-encrypt, ...
	\item Probleme in \inquotes{unerwarteten} Bereichen: Komprimierung, ...
\end{itemize}

-> Überleitung zu Tool

\end{mdframed}

In \cite{schubert11} wird das didaktischen Prinzip nach J.S. Brunner, wonach die Lehre sich \inquotes{in erster Linie an den Strukturen der zugrundeliegenden Wissenschaft orientieren soll}, erwähnt. Diese Strukturen werden auch Fundamentale Ideen genannt.

Hierbei handelt es sich um \inquotes{langlebige Konzepte}, die die \inquotes{Übertragung (Transfer) früher erworbener Kenntnisse auf [neue] Situationen} ermöglichen sollen (\cite{schubert11}). Insbesondere der nichtspezifische Transfer, also das Lernen von grundlegenden Begriffen, Prinzipien und Denkweisen, sollte an allgemeinbildenden Schulen und in der Hochschullehre im Gegensatz zum spezifischen Transfer, also dem geringfügigen Anpassen einer bekannten Situation an ein ähnliches Problem, im Vordergrund stehen.

Kennzeichnend für Fundamentale Ideen sind unter anderem das in \cite{schubert11} erwähnte Horizontalkriterium, die umfassende Anwendbarkeit oder Erkennbarkeit in vielen Bereichen einer Wissenschaft, und das Zeitkriterium, die längerfristige Relevanz der Idee. 

Als Empfehlung für die Hochschuldidaktik fassen die Autoren dieses Prinzip folgendermaßen zusammen:\\
\begin{quote}
\inquotes{Jeder Student wird im Laufe seines Berufslebens vermutlich mehreren Paradigmenwechseln der Informatik gegenüberstehen, wobei jeweils ein größerer Teil seines Wissens überflüssig oder fehlerhaft wird. Daher sollten die Fähigkeiten, die er während des Studiums erwirbt, möglichst robust gegenüber neuen wissenschaftlichen Entwicklungen sein und ihn befähigen,  Paradigmenwechsel  zu  bewältigen.  Folglich  müssen  Studenten  ein Bild von den dauerhaften Grundlagen, den Fundamentalen Ideen, Prinzipien, Methoden und Denkweisen der Informatik erlangen. Dazu sind in Vorlesungen und  Lehrbüchern  stets  die  Fundamentalen  Ideen,  die  sich  hinter  den  jeweils 
behandelten Sachgebieten verbergen, herauszuarbeiten, zu betonen, zu anderen Teilgebieten in Beziehung zu setzen und so in einen übergeordneten Zusammenhang  einzuordnen.} \cite{schubert11}
\end{quote}

Auch wenn das Prinzip Fundamentaler Ideen umfassender ist als das eingeschränkte Thema dieser Arbeit, so lässt sich doch eine klare Empfehlung für die Nutzung von TLS (oder anderer konkreter Protokolle oder Verfahren) in der Hochschullehre ableiten. Es sollten  insbesondere Prinzipien betrachtet werden, die als häufig verwendet und auch langfristig gültig für den Bereich der IT-Sicherheit, der Kryptographie oder der Sicherheitsprotokolle angesehen werden können.\\
Dieses Vorgehen wird von den Autoren auch in \cite{kluever12} als zentrale Aufgabe der Didaktik ausgemacht und als didaktische Reduktion, also \inquotes{die Rückführung komplexer Sachverhalte auf ihre wesentlichen  Elemente,  um  sie  für  Lernende  überschaubar  und  begreiflich  zu  machen}, bezeichnet.\\
Im Folgenden sollen einige Vorschläge für solche Prinzipien aufgeführt werden.

\subsection{Hybride Kryptosysteme}

In Kapitel \ref{cha_cryptographic_techniques} wurde bereits auf das Problem des Schlüsselaustausches bei symmetrischen Algorithmen zur Ver- und Entschlüsselung eingegangen. Durch asymmetrische Verfahren lässt sich dieses Problem leicht lösen. Diese Verfahren benötigen jedoch deutlich mehr Zeit als symmetrische für die Verarbeitung von Daten (je nach betrachteten Verfahren mindestens Faktor 1000). \\
Um die Vorteile beider Lösungen zu kombinieren, werden häufig hybride Kryptosysteme eingesetzt: Es wird für jede neue Kommunikation ein symmetrischer Schlüssel zufällig generiert (daher Sitzungsschlüssel genannt) und dem Kommunikationspartner mit dessen öffentlichen Schlüssel verschlüsselt zugesendet oder per Diffie-Hellman-Verfahren vereinbart (vgl. Abschnitt \ref{sec_diffie_hellman}). Nach der Entschlüsselung sind beide Partner im Besitz des gleichen Sitzungsschlüssels und können ihre Kommunikation symmetrisch verschlüsseln \cite{Schneier2006}.

Dieses Prinzip kommt neben seiner Anwendung in TLS unter anderem auch in IPSec und PGP zum Einsatz.

\subsection{Problem des Schlüsselaustausches und Authentifizierung}

Wie ebenfalls bereits in Kapitel \ref{cha_cryptographic_techniques} erwähnt, besteht bei der Nutzung von asymmetrischen Verfahren jedoch das Problem, die Identität des Besitzers des öffentlichen Schlüssels sicherzustellen.

Dieses Problem soll kurz erläutert werden. Möchte eine Person A eine Nachricht asymmetrisch verschlüsselt mit B austauschen, so benötigt sie den öffentlichen Schlüssel \(K^B_\text{public}\) von B. Ein Angreifer M, der die Kommunikation zwischen A und B lesen und verändern kann (ein sogenannter Man-in-the-Middle), kann diesen öffentlichen Schlüssel bei der Übermittlung durch seinen eigenen \(K^M_\text{public}\) austauschen. Die Nachricht von A kann er nun mit seinem geheimen Schlüssel entschlüsseln. Damit sein Angriff unbemerkt bleibt, kann er die Nachricht nun mit \(K^B_\text{public}\) verschlüsseln und an B senden.\todo{Bild}

Damit dieser Angriff erfolglos bleibt, muss es eine Möglichkeit geben, den Besitzer des Schlüssels zu verifizieren. Für diese Aufgabe wird in SSL/TLS auf Zertifikate und eine Public-Key-Infrastruktur (PKI) gesetzt. Ein Zertifikat enthält den öffentlichen Schlüssel und die Informationen des Besitzers bestehend aus dem Namen oder der URL des Servers und weiteren Angaben. Dieses Zertifikat wird von einer vertrauenswürdigen Instanz, der Certificate Authority (CA), nach der Überprüfung der Besitzeridentität mit ihrem geheimen Schlüssel signiert. Ein Empfänger des Zertifikats kann nun das Zertifikat mit dem öffentlichen Schlüssel der CA überprüfen, in dessen Besitz er im Vorwege sein muss\footnote{
	Heutige Browser und Betriebssysteme werden bereits mit Listen solcher CAs ausgeliefert.
}. Danach ist der Besitzer und der zugehörige öffentliche Schlüssel verifiziert und kann zum Verschlüsseln von Nachrichten genutzt werden\footnote{
	Die Schwächen, die in diesem System bestehen können, sind vielfältig, liegen jedoch nicht im Fokus dieser Arbeit. Erste Hinweise sind in Abschnitt \ref{sec_certificates} und in \cite{ferguson10} zu finden.
}. 

Die Nutzung von Zertifikaten und einer PKI findet unter anderem auch in IPSec oder S/MIME Verwendung. 

\subsection{Timing/Seitenkanal-Angriffe}
Bleichenbacher und weitere

\subsection{Replay-Angriffe}
Sequenznummer

\subsection{Krypto-Standardvorgehen und Probleme, wenn man dagegen verstößt}
Zufällige IVs in CBC, MAC-then-encrypt, ...

\subsection{Probleme in \inquotes{unerwarteten} Bereichen}
Komprimierung, ...

\section{Lernen durch Visualisierung/Exploration}

\begin{mdframed}
Erklärung best practices, ...

Vllt. auch E-Learning?

Hier die Überleitung zu Kapitel Implementierung. 
\end{mdframed}