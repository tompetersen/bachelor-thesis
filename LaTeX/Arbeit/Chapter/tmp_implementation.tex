\section{Architekturentscheidungen}

\begin{mdframed}
Zugrundeliegend: Gemeinsamkeit von Protokollen, Erweiterbarkeit/Pluginfähigkeit, ...

Analyse (Rückgriff auf Didaktik-Kapitel)

Warum Java? (UHH als Einstieg-> für alle verständlich und auch erweiterbar, ...)

\hrulefill{}

Welche TLS-Version? Oder SSL 2.0? Aus welchen Gründen? (1.2, aktuell, demonstriert Funktionsweise und Verhinderung von Angriffen; zum \inquotes{Durchklicken} von Angriffen wäre SSL schöner, ...)

Welche Dinge werden betrachtet, welche ausgelassen (Extensions, Zertifikatvalidierung, ...), aus welchen Gründen?

Ausgelassen:
\begin{itemize}
\item TLS-Extensions
\item Zertifikatsvalidierung, PKI und Verwandtes
\item Komprimierung
\item eingeschränkte Anzahl von CipherSuites (und Schlüsselaustausch)
\item Ältere Versionen
\item Clientauthentifizierung
\item Fragmentierung oder mehrere TlsMessages in einem Record
\item sichere Schlüsselverwaltung, RSA-Bleichenbacher (Noch implementieren?) 
\end{itemize}

\end{mdframed}


\section{Implementation notes}

\begin{mdframed}
Entwurf (UML ist toll,  ...), Implementierung, Tests, Evaluation...

Probleme/Schwierigkeiten bei der Umsetzung, ...

- Abstrakt \& Architektur: Automaten und States, ViewProvider, ...\\
- TLS: PDU-Struktur (insbesondere Fragment und CipherArten), Automaten (Security Parameters, Current/Pending States, ...), States (allgemein immer für Warten auf oder Received...), Crypto, Package-Struktur
\end{mdframed}
