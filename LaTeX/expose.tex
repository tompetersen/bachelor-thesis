%!TEX encoding = UTF-8 Unicode
\documentclass[
    12pt,
    headings=small,
    parskip=half,           % Ersetzt manuelles setzten von parskip/parindent.
    bibliography=totoc,
    numbers=noenddot,       % Entfernt den letzten Punkt der Kapitelnummern.
    open=any,               % Kapitel kann auf jeder Seite beginnen.
%   final                   % Entfernt alle todonotes und den Entwurfstempel.
    ]{scrreprt}

% ===================================Praeambel==================================

% Kodierung, Sprache, Patches {{{
\usepackage[T1]{fontenc}    % Ausgabekodierung; ermoeglicht Akzente und Umlaute
                            %  sowie korrekte Silbentrennung.
\usepackage[utf8]{inputenc} % Erlaub die direkte Eingabe spezieller Zeichen.
                            %  Utf8 muss die Eingabekodierung des Editors sein.
\usepackage[ngerman]{babel} % Deutsche Sprachanpassungen (z.B. Ueberschriften).
\usepackage{microtype}      % Optimale Randausrichtung und Skalierung.
\usepackage[
    babel,
    ]{csquotes}             % Korrekte Anfuehrungszeichen in der Literaturliste.
\usepackage{fixltx2e}       % Patches fuer LaTeX2e.
\usepackage{scrhack}        % Verhindert Warnungen mit aelteren Paketen.
% }}}

% Schriftarten {{{
\usepackage{mathptmx}       % Times. Package 'times.sty' is obsolete.
\usepackage[scaled=.92]{helvet}
\usepackage{courier}
% }}}

% Biblatex {{{
% \usepackage[
%     style=alphabetic,
%     backend=biber,
%     backref=true
%     ]{biblatex}             % Biblatex mit alphabetischem Style und biber.
% \bibliography{Literatur.bib}% Dateiname der bib-Datei.
% }}}

% Dokument- und Texteinstellungen {{{
\usepackage[
    a4paper,
    margin=2.54cm,
    marginparwidth=2.0cm,
    footskip=1.0cm
    ]{geometry}             % Ersetzt 'a4wide'.
\clubpenalty=10000          % Keine Einzelzeile am Beginn eines Paragraphen
                            %  (Schusterjungen).
\widowpenalty=10000         % Keine Einzelzeile am Ende eines Paragraphen
\displaywidowpenalty=10000  %  (Hurenkinder).
\usepackage{floatrow}       % Zentriert alle Floats.
\usepackage{ifdraft}        % Ermoeglicht \ifoptionfinal{true}{false}
\pagestyle{plain}           % keine Kopfzeilen
% \sloppy                     % großzügige Formatierungsweise
\deffootnote{1em}{1em}{\thefootnotemark.\ } % Verbessert Layout mehrzeiliger Fußnoten

\makeatletter
\AtBeginDocument{%
    \hypersetup{%
        pdftitle = {\@title},
        pdfauthor  = \@author,
    }
}
\makeatother
% }}}

% Weitere Pakete {{{
\usepackage{graphicx}       % Einfuegen von Graphiken.
\usepackage{tabu}           % Einfuegen von Tabellen.
\usepackage{multirow}       % Tabellenzeilen zusammenfassen.
\usepackage{multicol}       % Tabellenspalten zusammenfassen.
\usepackage{booktabs}       % Schönere Tabellen (\toprule\midrule\bottomrule).
\usepackage[nocut]{thmbox}  % Theorembox bspw. fuer Angreifermodell.
\usepackage{amsmath}        % Erweiterte Handhabung mathematischer Formeln.
\usepackage{amssymb}        % Erweiterte mathematische Symbole.
\usepackage{rotating}
\usepackage[
    printonlyused
    ]{acronym}              % Abkuerzungsverzeichnis.
\usepackage[
    colorinlistoftodos,
    textsize=tiny,          % Notizen und TODOs - mit der todonotes.sty von
    \ifoptionfinal{disable}{}%  Benjamin Kellermann ist das Package "changebar"
    ]{todonotes}            %  bereits integriert.
\usepackage[
    breaklinks,
    hidelinks,
    pdfdisplaydoctitle,
    pdfpagemode = {UseOutlines},
    pdfpagelabels,
    ]{hyperref}             % Sprungmarken im PDF. Laed das URL Paket.
    \urlstyle{rm}           % Entfernt die Formattierung von URLs.
\usepackage{breakurl}
\def\UrlBreaks{\do\/\do-}
\usepackage{listings}       % Spezielle Umgebung für...
    \lstset{                %  ...Quelltextformatierung.
        language=C,
        breaklines=true,
        breakatwhitespace=true,
        frame=L,
        captionpos=b,
        xleftmargin=6ex,
        tabsize=4,
        numbers=left,
        numberstyle=\ttfamily\footnotesize,
        basicstyle=\ttfamily\footnotesize,
        keywordstyle=\bfseries\color{green!50!black},
        commentstyle=\itshape\color{magenta!90!black},
        identifierstyle=\ttfamily,
        stringstyle=\color{orange!90!black},
        showstringspaces=false,
        }
% }}}

% ===================================Dokument===================================

\title{Bachelorarbeit über TLS (1.3)}
\author{Tom Petersen}
% \date{01.01.2015} % falls ein bestimmter Tag eingesetzt werden soll, einfach diese Zeile aktivieren

\begin{document}
\begin{titlepage}
\begin{center}\Large
	Universität Hamburg \par
	Fachbereich Informatik
	\vfill
	Exposé %TODO
	\vfill
	\makeatletter
	{\Large\textsf{\textbf{\@title}}\par}
	\makeatother
	\vfill
	vorgelegt von
	\par\bigskip
	\makeatletter
	{\@author} \par
	\makeatother
	geb. am 13. Dezember 1990 in Hannover \par
	Matrikelnummer 6359640 \par
	Studiengang Informatik
	\vfill
	\makeatletter
	eingereicht am {\@date}
	\makeatother
	\vfill
	%Betreuer: Dipl.-Inf. Heinz Mustermann \par
	%Erstgutachter: Prof. Dr.-Ing. Hannes Federrath \par
	%Zweitgutachter: N.N.
\end{center}
\ifoptionfinal{}{
\begin{tikzpicture}[remember picture, overlay]
    \node[draw, red, font=\ttfamily\bfseries\Huge, xshift=-50mm, yshift=238mm,
        rotate=10, text centered, text width=8cm, very thick, rounded
        corners=4mm] at (current page.south) {Entwurf vom \today};
\end{tikzpicture}}
\end{titlepage}


\chapter{SSL und TLS - ein Überblick}

SSL (Secure Socket Layer) bzw. TLS\footnote{Im weiteren Verlauf dieser Arbeit wird der Einfachheit lediglich von TLS gesprochen, es ist jedoch ebenso SSL gemeint. Bei etwaigen Unterschieden wird explizit auf diese eingegangen werden.} (Transport Layer Security) ist ein zustandsbehaftetes Protokoll, das auf dem TCP-Protokoll der Transportschicht des TCP/IP-Protokollstapels aufbaut\footnote{Es gibt auch DTLS (Datagram Transport Layer Security), ein auf TLS basierendes Protokoll, dass auch per UDP Daten übertragen kann}. Es bildet also eine Schicht zwischen Transport- und Anwendungsschicht. %, die von TLS jedoch weitgehend unbehelligt bleiben. 
Viele Protokolle der Anwendungsschicht nutzen TLS zur sicheren Datenübertragung, so beispielsweise HTTPS oder FTPS.

SSL wurde von der Firma Netscape entwickelt und nachdem es starke Verbreitung gefunden hatte, durch die IETF als TLS 1.0 in RFC 2246 standardisiert (TLS 1.0 entspricht hierbei SSL 3.1). Aktuell ist die TLS-Version 1.2 und an Version 1.3 wird gearbeitet. \cite{schmeh09}

Hauptaufgaben von TLS sind Authentifikation der Kommunikationspartner, symmetrische Verschlüsselung der Kommunikation sowie die Sicherstellung der Integrität der übertragenen Nachrichten. Die hierbei verwendeten kryptographischen Verfahren werden erst zu Beginn der Kommunikation festgelegt. 
\cite{eckert13}

\begin{description}
	\item[TLS 1.0] RFC 2246 - http://tools.ietf.org/html/rfc2246
	\item[TLS 1.1] RFC 4346 - http://tools.ietf.org/html/rfc4346
	\item[TLS 1.2] RFC 5246 - http://tools.ietf.org/html/rfc5246
	\item[TLS Extensions] RFC 3546 - http://tools.ietf.org/html/rfc3546, \\
RFC 3466 - http://tools.ietf.org/html/rfc4366, \\
RFC 6066 - http://tools.ietf.org/html/rfc6066
	\item[TLS 1.3] Draft - https://tools.ietf.org/html/draft-ietf-tls-tls13-05
\end{description}

\chapter{Funktionsweise und Teilprotokolle}

\todo{Zum Einstieg grobe Funktionsbeschreibung, evtl. schon mit Grafik über Verbindungsaufbau?, Grafik der TLS-Protokolle}

\section{Untere Schicht}

TLS besteht selbst aus zwei Schichten. In der unteren Schicht befindet sich das \emph{Record-Protokoll}, das die Daten von den Teilprotokollen der oberen Schicht entgegennimmt, diese Anwendungsdaten fragmentiert (maximale Paketgröße \(2^{14}\) Byte) und optional komprimiert. Danach wird je nach aktuell verhandelten krypographischen Funktionen die Integrität der Daten durch Berechnen und Anhängen eines MACs gesichert und die Nachricht verschlüsselt\footnote{Achtung: hier wird MAC-then-Encrypt angewendet. Laut \cite{AE2000} ist Encrypt-then-MAC vorzuziehen.}. \todo{Nach http://crypto.stackexchange.com/a/224 schlägt Schneier in Cryptography Engineering MAC-then-encrypt aus Gründen der Komplexheit von Encrypt-then-MAC vor. Nachlesen! Gibt es in der Informatik-Bibliothek: T FER 45399} 

Der Mac wird auf die folgende Weise berechnet\footnote{Entnommen aus \cite{eckert13}.}\todo{Ich vermute mal, dass das noch das alte Verfahren ist, das in TLS durch HMAC abgelöst wird. Nachschauen, ebenso SSLCompressed vmtl jetzt anders?}:
\begin{align*}
MAC = hash(MAC\_KEY | pad_2 |& \\
	hash(MAC\_KEY |& pad_1 | seq\_num | SSLCompressed.type |\\
	SSLCompre&ssed.length | SSLCompressed.fragment))
\end{align*}
wobei \(pad_1 = (0x36) \dots (0x36)\) (48 mal für MD5 und 40 mal für SHA) 
und \(pad_2 = (0x5c) \dots (0x5c)\) (48 mal für MD5 und 40 mal für SHA) gilt. Diese Werte entsprechen ipad und opad aus dem HMAC-Verfahren (\cite{hmac97}).

Der Record-Header enthält Informationen über die verwendete SSL/TLS-Version, den Content-Type (Handshake, Alert, ChangeCipherSpec, ApplicationData) und die Länge des Klartextfragments.

Aus dem nach Ausführung des Handshake-Protokolls (siehe nächsten Abschitt) beidseitig bekannten Master-Secret werden Schlüssel für die Erstellung des MACs sowie für die Kommunikation zwischen Client und Server berechnet (je einer pro Seite). Dazu werden solange Schlüsselblöcke nach dem folgenden Verfahren erstellt, bis alle Schlüssel konstruiert werden können: \todo{Wahrscheinlich auch das hier ein bisschen anders?}
\begin{align*}
key\_block = &MD5(ms | SHA('A' | ms | R_C | R_S)) |\\
	&MD5(ms | SHA('BB' | ms | R_C | R_S)) |\\
 	&MD5(ms | SHA('CCC' | ms | R_C | R_S))
\end{align*}

\section{Obere Schicht}

In der oberen Schicht sind vier Teilprotokolle spezifiziert: \emph{Handshake-, ChangeCipherSpec-, Alert-} und \emph{ApplicationData-Protokoll}.

Das \emph{Handshake-Protokoll} dient zur Vereinbarung kryptographischer Verfahren und zur Aushandlung eines Schlüssels für die symmetrische Verschlüsselung der später gesendeten Daten. Der Handshake kann folgendermaßen ablaufen (abhängig von optionalem Clientzertifikat, Preshared Key, erneute Verbindung, ...; entnommen aus \cite{eckert13}): 

\todo{als ordentliche Grafik ohne Clientverifikation}

\begin{itemize}
\item --> client\_hello mit \\
\(H_C = (\)4 Byte Zeitstempel, 28 Byte Zufallszahl, Sitzungsidentifikator (!= Null bei bereits existierender Sitzung), Cipher Suite\()\)

\item <-- server\_hello mit \\
\(H_S = (\)4 Byte Zeitstempel, 28 Byte Zufallszahl, Sitzungsidentifikator (falls vom Client gewünschte Sitzung bedient werden kann), Cipher Suite, die vom Server unterstützt wird\()\)

\item <-- Server Zertifikat (inklusive öffentlichem Schlüssel des Servers, meist nach X.509v3)

\item Client: Zertifikatverifikation

\item --> Generierung und Senden von PreMasterSecret (48 Byte) verschlüsselt mit öffent. Schlüssel des Servers (bei RSA) oder Diffie-Hellman-Verfahren

\item Client und Server: aus Zufallszahl des Clients \(R_C \text{ aus } H_C\), Zufallszahl des Servers \(R_S \text{ aus } H_S\) und PreMasterSecret wird das MasterSecret berechnet\todo{so wahrscheinlich < TLS 1.2, danach Pseudozufallsfunktion -> Nachschauen}
\begin{align*}
master\_secret = &MD5(Pre | SHA('A' | Pre | R_C | R_S)) |\\
	&MD5(Pre | SHA('BB' | Pre | R_C | R_S)) |\\
 	&MD5(Pre | SHA('CCC' | Pre | R_C | R_S))
\end{align*}

\item --> Handshakeabschluss mit MD5 und SHA-MAC über alle bisher ausgetauschten Nachrichten

\item <-- Handshakeabschluss mit MD5 und SHA-MAC über alle bisher ausgetauschten Nachrichten
\end{itemize}


Das \emph{ChangeCipherSpec-Protokoll} dient dazu, die vereinbarten kryptographischen Verfahren zu ändern. Es enthält lediglich eine Nachricht mit dem Wert 1, die für das Übernehmen der während des Handshakes ausgehandelten Verfahren steht. \cite{eckert13}

Das \emph{Alert-Protokoll} dient dazu, auftretende Fehler oder Warnungen zu versenden, die während des Datenaustausches auftreten. \todo{Übersicht}

Das \emph{ApplicationData-Protokoll} ist zuständig für das Durchreichen von Anwendungsdaten, die von der Anwendungsschicht gesendet werden sollen.\cite{Schneier2006}

\section{Sitzungs- und Verbindungskonzept}

TLS erstellt beim ersten Handshake eine Sitzung zwischen Client und Server. Hierbei wird ein Sitzungsidentifikator erstellt, der beim server\_hello mitgesendet wird. Weiterhin wird sich in der Sitzung das Zertifikat des Gegenübers, optional das Kompressionsverfahren, die CipherSpec und das MasterSecret gemerkt.

Ein Client kann nun, wenn er den Sitzungsidentifikator beim client\_hello mitschickt, eine alte Sitzung in Form einer neuen Verbindung wiederaufnehmen oder mehrere Verbindungen parallel aufbauen. Eine Verbindung wird dabei durch Client- und Server-Zufallszahlen \(R_C\) und \(R_S\), die generierten Schlüssel, einen Initialisierungsvektor \todo{Wie wird sich auf den geeinigt? Übertragen? Aus master-secret?} sowie aktuelle Sequenznummern beschrieben.

Beim Verbindungsaufbau kann so ein verkürzter Handshake genutzt werden, bei dem weniger Nachrichten gesendet werden müssen. Es kann dabei auf Neuberechnung des master-secrets, Server- und Client-Validierung und Aushandlung der CipherSpec verzichtet werden.\todo{Grafik des verkürzten Handhakes}

\section{Ciphersuites}



\chapter{Angriffe}

\section{}

\bibliography{quellen}
\bibliographystyle{myalpha}

\end{document}
