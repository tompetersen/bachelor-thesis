\input{preambel.tex}

% =============================
% = Ab hier Inhalte ändern... 
% =============================

\title{Working principle, attacks\\ and defenses of SSL/TLS}
\author[Tom Petersen]{Tom Petersen}
\institute[Uni Hamburg]{University of Hamburg\\ Department of Informatics}
\date{}%12.01.2016}

\begin{document}

\begingroup
	\setbeamertemplate{background canvas}[titlepage]
	\begin{frame}[plain]
		\vskip8mm
		\includegraphics[width=2.2cm]{pic/svs_logo_hires-ohne-was.png}
		 \vskip-20mm % dies geht nur bei kurzen Vortragstiteln
		\titlepage
		\vspace{\fill}
		\includegraphics[width=2.9cm]{pic/UHH-Logo_2010_Farbe_RGB_hires_nomargin.png}
		\vskip20pt
	\end{frame}
\endgroup

%\begin{frame}{Agenda}
%	\tableofcontents
%\end{frame}

\section{Motivation}
%- brief overview (usage, historical version history)
%- working principle 
%	- message overview image from RFC, subprotocols
%	- key material
%	- cipher suites

\begin{frame}{Motivation TLS}

\begin{itemize}
\item security protocol above the transport layer of the OSI model

\item provides authentication of communication partners, encryption and integrity check of sent messages

\item used by many higher layer protocols and applications 

\item SSL/TLS is the most used security protocol today. \footnote{[Jörg Schwenk. Sicherheit und Kryptographie im Internet. 4. Auflage Springer Vieweg, 2014.}
\end{itemize}

%- developed as a part of netscape navigator in the early 90s (1994) as SSL (v1, v2, v3) and was soon used in other browsers and applications
%
%- 1999 IETF standardized it as TLS 1.0
%
%- current 1.2, 1.3 in progress

\end{frame}

\section{TLS overview}

\begin{frame}[c]{Protocol hierarchie}
	\begin{figure}[H]
		\centering
		\begin{tikzpicture}[node distance=0cm,outer sep = 0pt]
			\tikzstyle{protocol}=[draw, rectangle, minimum height=1cm, minimum width=3.5cm, fill=blue!15, anchor=south west]
			\tikzstyle{row}=[draw, rectangle, minimum height=1cm, minimum width=14cm, anchor=south west]
	
			\node[protocol] (handshake) at (0,2) {Handshake};
			\node[protocol] (change) [right = of handshake] {ChangeCipherSpec};
			\node[protocol] (alert) [right = of change] {Alert};
			\node[protocol] (application) [right = of alert] {ApplicationData};
	
			\node[row, fill=gray!20,] (record) at (0,1) {Record Protocol};
			\node[row] (tcp) at (0,-0.5) {Transport Layer};
		\end{tikzpicture}
	\end{figure}
\end{frame}


\section{University teaching}
%- briefly: reasons for using TLS in university teaching
%- fundamental ideas/ didactic reduction
%
%- advantages of using exploration/simulations in teaching ->

\begin{frame}{Fundamental ideas}

	\begin{block}{\textbf{Fundamental ideas}}
	long-lasting principles, which are applicable or observable in multiple contexts.
		\begin{flushright}
		\tiny \color{svsgrau2}Andreas Schwill. Computer Science Education based on Fundamental Ideas. \textit{http://ddi.uni-muenster.de/didaktik/Forschung/Israel97.pdf} (Retrieved 07.1.2016)
		\end{flushright}
	\end{block}
	
	Examples:

	\pause

	\begin{itemize}
		\item hybrid cryptosystems
		\pause
		\item authenticated key exchange
		\pause
		\item side-channel attacks
		\pause
		\item cryptographic principles, e.g. unpredictable IVs
	\end{itemize}
\end{frame}

\begin{frame}{Discovery learning with simulations}
	
	\begin{block}{\textbf{Discovery learning} (also \textbf{exploratory learning})}
	learning by interacting with the world by exploring and manipulating objects and performing experiments.
		\begin{flushright}
		\tiny \color{svsgrau2} J. S. Bruner. On Knowing: Essays for the left hand. Harvard University press, 1979.
		\end{flushright}
	\end{block}

\vspace{1cm}

	\begin{block}{\textbf{Simulations}}
	interactive computer programs modelling activities, which enable us to observe normally hidden processes.
		\begin{flushright}
		\tiny  \color{svsgrau2} Helmut M. Niegemann et al. Kompendium multimediales Lernen. Springer, 2008.
		\end{flushright}
	\end{block}

\end{frame}

\section{Simulating protocols}

%protocol.edu
%	- TLS plugin
%	- extensions/plugins

%\begin{frame} {Developed application}
%- simulating TLS
%
%- extendable (requirement) -> with this in mind i developed the application plugin-ready 
%
%- general application for simulating two party protocol flows (usually client and server)
%
%- observable internal party states to understand the processes happening during a protocol flow
%
%- interactive to see consequences of messages or changes of messages
%\end{frame}

\begin{frame}[c] {Live demo}

	\begin{center}
			\includegraphics[width=8cm]{pic/ScreenshotTLS.png}
	\end{center}

\end{frame}

\begin{frame}{Possible next steps}

	\begin{itemize}
		\item Unimplemented in TLS plugin
			\begin{itemize}
			\item certificate validation
			\item session resumption
			\item client authentication
			\item TLS extensions
			\end{itemize}
		\item Implement other protocols
	\end{itemize}

\end{frame}

\begin{frame}[c]{}
\begin{center}
\LARGE Thank you!
\end{center}
\end{frame}


\end{document}