\chapter{TLS in der Lehre}

Stichworte

\begin{itemize}
	\item "'einfaches"' Protokoll, das gut zur Erklärung einzelner Verfahren dienen kann (Record-Protocol für vertrauliche und authentifizierte Nachrichtenübertragung, Handshake zur Schlüsselaushandlung, ...)
	\item Schwerpunkte für die Lehre setzen
	\item Viele Angriffe, die Abwehrmaßnahmen und auch "'Prinzip des schwächsten Kettenglieds"' deutlich machen (CBC-IV, Padding, MAC-then-Encrypt)
	\item weite Verbreitung -> notwendiges Verständnis
	\item Familien von Angriffen betrachten?
\end{itemize}

Informatik-Didaktik, Anforderungen an Visualisierungen, ... -> Methodik erläutern

\begin{itemize}
\item Ludger Humbert, Didaktik der Informatik, Stuttgart: Teubner, 2005
\item Sigried Schubert/Andreas Schwill, Didaktik der Informatik, Heidelberg/Berlin: Spektrum, 2004
\item Hilbert Meyer, Unterrichtsmethoden, 2 Bände, Berlin: Cornelsen, 11.Aufl. 2005
\end{itemize}

Vllt. auch E-Learning?

Hier die Überleitung zu Kapitel Implementierung. 