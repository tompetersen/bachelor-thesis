\input{preambel.tex}

% =============================
% = Ab hier Inhalte ändern... 
% =============================

\title{Working principle, attacks\\ and defenses of SSL/TLS}
\author[Tom Petersen]{Tom Petersen}
\institute[Uni Hamburg]{University of Hamburg\\ Department of Informatics}
\date{}%12.01.2016}

\begin{document}

\begingroup
	\setbeamertemplate{background canvas}[titlepage]
	\begin{frame}[plain]
		\vskip8mm
		\includegraphics[width=2.2cm]{pic/svs_logo_hires-ohne-was.png}
		 \vskip-20mm % dies geht nur bei kurzen Vortragstiteln
		\titlepage
		\vspace{\fill}
		\includegraphics[width=2.9cm]{pic/UHH-Logo_2010_Farbe_RGB_hires_nomargin.png}
		\vskip20pt
	\end{frame}
\endgroup

\begin{frame}{Agenda}
	\tableofcontents
\end{frame}

\section{Transport Layer Security}
%- brief overview (usage, historical version history)
%- working principle 
%	- message overview image from RFC, subprotocols
%	- key material
%	- cipher suites

\begin{frame}{TLS Overview}

- authentication of communication partners, encryption and integrity check of sent messages

- widely used in HTTPS, SFTP, SMTPS, ... 

- developed as a part of netscape navigator in the early 90s (1994) as SSL (v1, v2, v3) and was soon used in other browsers and applications

- 1999 IETF standardized it as TLS 1.0

- current 1.2, 1.3 in progress

\end{frame}

\begin{frame}[c]{Protocol hierarchie}

	\begin{figure}[H]
		\centering
		\begin{tikzpicture}[node distance=0cm,outer sep = 0pt]
			\tikzstyle{protocol}=[draw, rectangle, minimum height=1cm, fill=blue!15, anchor=south west]
			\tikzstyle{row}=[draw, rectangle, minimum height=1cm, minimum width=10cm, anchor=south west]
	
			\node[protocol] (handshake) at (0,2) {Handshake};
			\node[protocol] (change) [right = of handshake] {Change Cipher Spec};
			\node[protocol] (alert) [right = of change, minimum width=1.38cm] {Alert};
			\node[protocol] (application) [right = of alert] {Application Data};
	
			\node[row, fill=gray!20,] (record) at (0,1) {Record Protocol};
			\node[row] (tcp) at (0,-0.5) {TCP};
		\end{tikzpicture}
	\end{figure}

\end{frame}

\begin{frame}[fragile, c]{Message flow}

%TODO: Vllt auf "Standardfall" wie im Programm einschränken
\lstset{
	style=default,
	frame=none
}
\begin{figure}[H]
\centering
\begin{lstlisting}
§\textbf{Client}§                                       §\textbf{Server}§

ClientHello        -------->
                                      ServerHello
                               ServerCertificate*
                               ServerKeyExchange*
                   <--------      ServerHelloDone          
ClientKeyExchange
[ChangeCipherSpec]
Finished           -------->
                               [ChangeCipherSpec]
                   <--------             Finished
[Application Data] <------->   [Application Data]
\end{lstlisting}
\end{figure}

\begin{flushright}
\tiny \textit{Based on the TLS 1.2 specification, RFC 5246.}
\end{flushright}

\end{frame}

\begin{frame}[c]{Cipher suites}
\centering
TLS\_{\color{svshellblau2}RSA}\_WITH\_{\color{svsrot}AES\_128\_CBC}\_{\color{svsgrau2}SHA}

\vspace{1cm}

TLS\_{\color{svshellblau2}DHE\_RSA}\_WITH\_{\color{svsrot}AES\_128\_GCM}\_{\color{svsgrau2}SHA256}
\end{frame}

\section{University teaching}
%- briefly: reasons for using TLS in university teaching
%- fundamental ideas/ didactic reduction
%
%- advantages of using exploration/simulations in teaching ->

\begin{frame}{Didactic reduction (?)}
- concrete methods (like protocols) should be abstracted

- long-lasting principles, which are used in multiple fields

\begin{itemize}
	\item hybrid cryptosystems 
	\item authenticated key exchange
	\item side-channel attacks
	\item cryptographic principles
\end{itemize}

\end{frame}

\begin{frame}{Explorative learning}

- discovering and studying a topic by oneself

- useful for complex topics, which are hard to understand with other learning materials

- requires appropriate software: often simulations are used = interactive computer programs modelling activities, which enable us to observe normally hidden processes
 
\end{frame}

\section{Simulating protocols}

%protocol.edu
%	- TLS plugin
%	- extensions/plugins

\begin{frame} {Developed application}
- application for simulating two party protocol flows

- extendable (requirement)

- TLS plugin

	- not implemented in TLS plugin (?) p. 39  
\end{frame}

\begin{frame} {Live demo}
- cipher suite chosing

- server/client views

- start connection 

- message and message details view

- info view

- finish handshake

- edit bytes and watch occuring error

- echo plugin (?)

\end{frame}

\begin{frame}{Conclusion}
whatever
\end{frame}

\end{document}