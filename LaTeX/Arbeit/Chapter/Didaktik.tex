\chapter{TLS in der Lehre}

\section{Warum eignet sich TLS bzw. warum sollte man sich überhaupt damit befassen?}

\begin{mdframed}
\begin{itemize}
	\item "'einfaches"' Protokoll, das gut zur Erklärung einzelner Verfahren dienen kann (Record-Protocol für vertrauliche und authentifizierte Nachrichtenübertragung, Handshake zur Schlüsselaushandlung, ...)
	\item Viele Angriffe, die Abwehrmaßnahmen und auch "'Prinzip des schwächsten Kettenglieds"' deutlich machen (CBC-IV, Padding, MAC-then-Encrypt)
	\item weite Verbreitung -> notwendiges Verständnis
\end{itemize}
\end{mdframed}

\section{Fundamentale Ideen / Schwerpunkte für die Lehre}

\begin{mdframed}

Was ist das? -> Im kleineren Rahmen für TLS (wahrscheinlich nicht ganz passend, aber es geht um die Richtung)

Was könnte hier erwähnt werden?
\begin{itemize}
	\item Hybride Kryptosysteme (Symmetrisch und asymmetrisch)
	\item Problem des Schlüsselaustausches und Authentifizierung
	\item Familien von Angriffen betrachten?\\
	- Timing/Seitenkanal-Angriffe: Bleichenbacher und weitere\\
	- Replay-Angriffe: Sequenznummer
	\item Krypto-Standardvorgehen und Probleme, wenn man dagegen verstößt: Zufällige IVs in CBC, MAC-then-encrypt, ...
	\item Probleme in \inquotes{unerwarteten} Bereichen: Komprimierung, ...
\end{itemize}

-> Überleitung zu Tool

\end{mdframed}

In \cite{schubert11} wird das didaktischen Prinzip nach J.S. Brunner, wonach die Lehre sich \inquotes{in erster Linie an den Strukturen der zugrundeliegenden Wissenschaft orientieren soll}, erwähnt. Diese Strukturen werden auch Fundamentale Ideen genannt.

Hierbei handelt es sich um \inquotes{langlebige Konzepte}, die die \inquotes{Übertragung (Transfer) früher erworbener Kenntnisse auf [neue] Situationen} ermöglichen sollen (\cite{schubert11}). Insbesondere der nichtspezifische Transfer, also das Lernen von grundlegenden Begriffen, Prinzipien und Denkweisen, sollte an allgemeinbildenden Schulen und in der Hochschullehre im Gegensatz zum spezifischen Transfer, also dem geringfügigen Anpassen einer bekannten Situation an ein ähnliches Problem, im Vordergrund stehen.

Kennzeichnend für Fundamentale Ideen sind unter anderem das in \cite{schubert11} erwähnte Horizontalkriterium, die umfassende Anwendbarkeit oder Erkennbarkeit in vielen Bereichen einer Wissenschaft, und das Zeitkriterium, die längerfristige Relevanz der Idee. 

Als Empfehlung für die Hochschuldidaktik fassen die Autoren dieses Prinzip folgendermaßen zusammen:\\
\begin{quote}
\inquotes{Jeder Student wird im Laufe seines Berufslebens vermutlich mehreren Paradigmenwechseln der Informatik gegenüberstehen, wobei jeweils ein größerer Teil seines Wissens überflüssig oder fehlerhaft wird. Daher sollten die Fähigkeiten, die er während des Studiums erwirbt, möglichst robust gegenüber neuen wissenschaftlichen Entwicklungen sein und ihn befähigen,  Paradigmenwechsel  zu  bewältigen.  Folglich  müssen  Studenten  ein Bild von den dauerhaften Grundlagen, den Fundamentalen Ideen, Prinzipien, Methoden und Denkweisen der Informatik erlangen. Dazu sind in Vorlesungen und  Lehrbüchern  stets  die  Fundamentalen  Ideen,  die  sich  hinter  den  jeweils 
behandelten Sachgebieten verbergen, herauszuarbeiten, zu betonen, zu anderen Teilgebieten in Beziehung zu setzen und so in einen übergeordneten Zusammenhang  einzuordnen.} \cite{schubert11}
\end{quote}

Auch wenn das Prinzip Fundamentaler Ideen umfassender ist als das eingeschränkte Thema dieser Arbeit, so lässt sich doch eine klare Empfehlung für die Nutzung von TLS (oder anderer konkreter Protokolle oder Verfahren) in der Hochschullehre ableiten. Es sollten  insbesondere Prinzipien betrachtet werden, die als häufig verwendet und auch langfristig gültig für den Bereich der IT-Sicherheit, der Kryptographie oder der Sicherheitsprotokolle angesehen werden können.\\
Dieses Vorgehen wird von den Autoren auch in \cite{kluever12} als zentrale Aufgabe der Didaktik ausgemacht und als didaktische Reduktion, also \inquotes{die Rückführung komplexer Sachverhalte auf ihre wesentlichen  Elemente,  um  sie  für  Lernende  überschaubar  und  begreiflich  zu  machen}, bezeichnet.\\
Im Folgenden sollen einige Vorschläge für solche Prinzipien aufgeführt werden.

\subsection{Hybride Kryptosysteme}

In Kapitel \ref{cha_cryptographic_techniques} wurde bereits auf das Problem des Schlüsselaustausches bei symmetrischen Algorithmen zur Ver- und Entschlüsselung eingegangen. Durch asymmetrische Verfahren lässt sich dieses Problem leicht lösen. Diese Verfahren benötigen jedoch deutlich mehr Zeit als symmetrische für die Verarbeitung von Daten (je nach betrachteten Verfahren mindestens Faktor 1000). \\
Um die Vorteile beider Lösungen zu kombinieren, werden häufig hybride Kryptosysteme eingesetzt: Es wird für jede neue Kommunikation ein symmetrischer Schlüssel zufällig generiert (daher Sitzungsschlüssel genannt) und dem Kommunikationspartner mit dessen öffentlichen Schlüssel verschlüsselt zugesendet oder per Diffie-Hellman-Verfahren vereinbart (vgl. Abschnitt \ref{sec_diffie_hellman}). Nach der Entschlüsselung sind beide Partner im Besitz des gleichen Sitzungsschlüssels und können ihre Kommunikation symmetrisch verschlüsseln \cite{Schneier2006}.

Dieses Prinzip kommt neben seiner Anwendung in TLS unter anderem auch in IPSec und PGP zum Einsatz.

\subsection{Problem des Schlüsselaustausches und Authentifizierung}

Wie ebenfalls bereits in Kapitel \ref{cha_cryptographic_techniques} erwähnt, besteht bei der Nutzung von asymmetrischen Verfahren jedoch das Problem, die Identität des Besitzers des öffentlichen Schlüssels sicherzustellen.

Dieses Problem soll kurz erläutert werden. Möchte eine Person A eine Nachricht asymmetrisch verschlüsselt mit B austauschen, so benötigt sie den öffentlichen Schlüssel \(K^B_\text{public}\) von B. Ein Angreifer M, der die Kommunikation zwischen A und B lesen und verändern kann (ein sogenannter Man-in-the-Middle), kann diesen öffentlichen Schlüssel bei der Übermittlung durch seinen eigenen \(K^M_\text{public}\) austauschen. Die Nachricht von A kann er nun mit seinem geheimen Schlüssel entschlüsseln. Damit sein Angriff unbemerkt bleibt, kann er die Nachricht nun mit \(K^B_\text{public}\) verschlüsseln und an B senden.\todo{Bild}

Damit dieser Angriff erfolglos bleibt, muss es eine Möglichkeit geben, den Besitzer des Schlüssels zu verifizieren. Für diese Aufgabe wird in SSL/TLS auf Zertifikate und eine Public-Key-Infrastruktur (PKI) gesetzt. Ein Zertifikat enthält den öffentlichen Schlüssel und die Informationen des Besitzers bestehend aus dem Namen oder der URL des Servers und weiteren Angaben. Dieses Zertifikat wird von einer vertrauenswürdigen Instanz, der Certificate Authority (CA), nach der Überprüfung der Besitzeridentität mit ihrem geheimen Schlüssel signiert. Ein Empfänger des Zertifikats kann nun das Zertifikat mit dem öffentlichen Schlüssel der CA überprüfen, in dessen Besitz er im Vorwege sein muss\footnote{
	Heutige Browser und Betriebssysteme werden bereits mit Listen solcher CAs ausgeliefert.
}. Danach ist der Besitzer und der zugehörige öffentliche Schlüssel verifiziert und kann zum Verschlüsseln von Nachrichten genutzt werden\footnote{
	Die Schwächen, die in diesem System bestehen können, sind vielfältig, liegen jedoch nicht im Fokus dieser Arbeit. Erste Hinweise sind in Abschnitt \ref{sec_certificates} und in \cite{ferguson10} zu finden.
}. 

Die Nutzung von Zertifikaten und einer PKI findet unter anderem auch in IPSec oder S/MIME Verwendung. 

\subsection{Seitenkanalangriffe}
Seitenkanalangriffe sind Angriffe, die nicht das kryptographische Verfahren direkt angreifen, sondern versuchen Informationen über die Nachricht oder verwendete Schlüssel aus anderen Kanälen zu erhalten. Hierbei kann es sich beispielsweise um Zeit- oder Stromverbrauchsmessung handeln, aber auch um Messung von elektromagnetischer Abstrahlung, Schall oder sogar des Erdungspotentials eines Rechners. 

Wenn sich unterschiedliche Schlüssel- oder Nachrichtenbits verschieden auf die Werte des Seitenkanals auswirken, so lassen sich Schlüssel bzw. Nachricht oftmals aus den gemessenen Informationen ableiten. Gerade bei einfachen Geräten wie Smartcards können schon einfache Angriffe erfolgreich sein.

Um vor Seitenkanalangriffen geschützt zu sein, muss eine Implementierung sich für unterschiedliche Eingaben möglichst gleich verhalten. Ein Beispiel als Maßnahme gegen Timing-Angriffe (also die Messung der benötigten Zeit einer Operation) ist die Einführung einer konstanten Zeit für die Ausführungsdauer. Dazu wird eine Zeit \(d\) gewählt, die länger ist als jede mögliche benötigte Ausführungsdauer \(t\) der Operation. Nach der Ausführung wird immer die Zeit \(d-t\) gewartet, sodass die Operation insgesamt die konstante Zeit \(d\) benötigt. Diese Möglichkeit schützt natürlich nicht vor anderen Seitenkanalangriffen, für deren Abwehr andere Schutzmaßnahmen getroffen werden müssen \cite{ferguson10}.

Seitenkanalangriffe in Form von Timing-Angriffen werden auch bei einigen Angriffen gegen SSL/TLS eingesetzt, wie beispielsweise bei dem Bleichenbacher- (Abschnitt \ref{sec_attack_bleichenbacher}) oder dem Padding-Orakel-Angriff (Abschnitt \ref{sec_attack_padding_oracle}).

\subsection{Replay-Angriffe}

Bei einem Replay-Angriff sendet ein Angreifer zuvor aufgenommene Nachrichten erneut an den Empfänger. Da es sich um eine echte Nachricht handelt, müssen Gegenmaßnahmen getroffen werden, um dieses erneute Senden zu erkennen. Möglichkeiten hierfür sind beispielsweise die Einführung von Sequenznummern wie bei TLS, die Nutzung von Zeitstempeln oder ein Challenge-Response-Verfahren, dass einem Sender eine zufällige Aufgabe stellt, deren Lösung er in der Nachricht mitsenden muss \cite{ferguson10}.

\subsection{Krypto-Standardvorgehen und Probleme, wenn man dagegen verstößt}
Zufällige IVs in CBC, MAC-then-encrypt, ...

\subsection{Probleme in \inquotes{unerwarteten} Bereichen}
Komprimierung, ...

\section{Lernen durch Visualisierung/Exploration}

\begin{mdframed}
Erklärung best practices, ...

Vllt. auch E-Learning?

Hier die Überleitung zu Kapitel Implementierung. 
\end{mdframed}

\textbf{Didaktik der Informatik}

p.138
 Die Arbeit mit einem Ex-
plorationsmodul wird motiviert aus dem Wunsch des Schülers, mit Elementen 
zu  agieren,  die  aufgrund  ihrer  Abstraktion  oder  Komplexität  mit  anderen 
Lernmaterialien schwerer erfassbar oder schwerer darstellbar sind. 

. Allerdings benö-
tigt der Schüler geeignete Software für seine Entdeckungsreisen im Sinne des 
explorativen  Lernens.  Wir  bezeichnen  diese  spezielle  Software  als  Explorati-
onsbausteine  oder  allgemeiner  als  lernförderliche  Software (Brinda/Schubert, 
2002b).  Ein  einfaches  Beispiel  dafür  ist  das  Anfordern  einer  Webseite.  Die 
Wirkprinzipien bleiben den Schülern verborgen, wenn sie einen Browser ver-
wenden. 

Es steht eine Lernsoftware zur Verfügung, mit der diese verborgenen Prozesse 
erkundet werden können (Steinkamp, 1999). 
 
Die  Schüler  können  das  graphisch  dargestellte  Netzwerk  manipulieren  (Abb. 
5.5) und die ausgeführten Protokollschritte beobachten (Abb. 5.6). 

Sie überprüfen auf diese Weise ihre Hypothesen zu Aufbau und Funktionswei-
se von Rechnernetzen (vgl. Abschnitt 8.4). Mehr Möglichkeiten für das explora-
tive  Lernen  liefert  die  „Lernumgebung  für  objektorientiertes  Modellieren 
(LeO)“ (vgl. Abschnitt 8.5). 

p.201
Unterricht soll anschaulich sein. Unterrichtshilfen sollen die Veranschaulichung 
unterstützen. Für diese Maxime gibt es in der Historie zahlreiche Begründun-
gen und Belege. 

p.204
Für Michael (1983) dienen Unterrichtshilfen der Veranschaulichung von Reali-
tät, wobei folgende Funktionen bedeutsam sind: 
1.  Veranschaulichung als Motivationshilfe, um Lernprozesse in Gang zu set-
zen oder zu halten; 
2.  Veranschaulichung  als  Erkenntnishilfe,  um  Lernprozesse  zu  erleichtern, 
besseres Verstehen zu ermöglichen; 
3.  Veranschaulichung als Reproduktionshilfe, um Gelerntes intensiver einzu-
prägen und genauer wiedergeben zu können. 

p.208
Experimente im Informatikunterricht 

Das Experimentieren ist eine wichtige Erkenntnismethode, die in jedem Unter-
richtsfach angewendet werden kann:  

„Heute  versteht  man  unter  einem  Experiment  einen  planmäßigen  und  kontrollierten 
Versuch  zur  Überprüfung  einer  Fragestellung  oder  zur  Aufklärung  eines  unklaren 
Sachverhalts“ (Meyer, 1994, S. 313). 

Meyer  unterscheidet  drei  Arten  von  Experimenten:  Forschungsexperimente, 
Unterrichtsexperimente und freies Experimentieren bzw. Tüfteln, Erproben. In 
der Informatik sind alle drei Arten sinnvoll. 

p.209
  Typische  Experimente  in  der  Informatik  sind:  Tests  (Testen), 
Messreihen,  Fallstudien,  Feldbeobachtungen  und  kontrollierte  Experimente.“ 

 Wir empfehlen Experimente für das Erlernen 
der Wirkprinzipien von Informatiksystemen (vgl. Kapitel 9), z.B. von Rechner-
netzen. [...]
Außerdem sind die ablaufenden Prozesse und Protokolle 
im  System  verborgen.  Ideal  wären  handliche  Modellnetzwerkkomponenten 
(Server, Client, Router, Gateway, Bridge, Verbindungsleitungen und -stecker), 
die es erlauben, bei reduzierter Komplexität die vom verteilten System auszu-
führenden  Protokollschritte  zu  beobachten.  Vorerst  fehlen  gegenständliche 
Unterrichtsmittel für solche Experimente. Deshalb wurden Experimentierum-
gebungen in Form von Informatiksystemen entwickelt.  

Dabei  kommt  es  darauf  an,  dass  die  wesentlichen  Schülertätigkeiten  möglich 
sind: Planen des Experiments, Manipulieren und Beobachten des Untersuchungsge-
genstandes.  [...]  Immer  ist  eine  besondere  Ausgabe-
funktionalität erforderlich, die die verborgenen Prozesse, hier die ausgeführten 
Protokollschritte, an die Benutzungsoberfläche bringt. Eine solche „didaktische 
Trace-Komponente“ (trace – Spur) kann den Zustand, den Datenfluss oder die 
Struktur eines Systems aufdecken. Die Funktionsweise der Rechnernetze wird 
in der Experimentierumgebung nachgebildet.

p.213-214
Didaktik der Informatik und E-Learning 

Forschungsschwerpunkten: unter anderem\\
x Entwicklung von Explorationsmodulen zur Unterstützung von: 
- Sichtenwechsel auf Lerngegenstand, 
- Konstruktion von Lösungen, 
- Verknüpfen von Lösungselementen, 
- Bewerten von Modellen, 
x Entwicklung von Software-Experimenten. 

Ein solcher Standardfall ist der Sichtenwechsel auf den Lerngegenstand. Im Informa-
tikunterricht gelingt die Entwicklung eines Klassendiagramms für den objekt-
orientierten Entwurf, z.B. einer Bibliotheksverwaltung, vielen Schülern leichter, 
wenn sie in diesem Abstraktionsprozess zur Sicht auf das anschaulichere Ob-
jektdiagramm wechseln können, das wesentlich näher am Anwendungsszenario 
ist (Abb. 7.2 und Abb. 7.3). [...] Hier liegt eine Stärke des Lernens mit Informatiksyste-
men. Deshalb ist Bildungssoftware für den Sichtenwechsel prinzipiell empfeh-
lenswert. 
Einen zweiten Standardfall bildet die Konstruktion von Lösungen (s. auch Kapitel 
6). In der Informatik wird z.B. die Neustrukturierung eines objektorientierten 
Entwurfes  erleichtert,  wenn  man  das  Klassendiagramm  editieren,  d.h.  leicht 
modifizieren,  kann.  Im  Konstruktionsprozess  steht  häufig  das  Verknüpfen  von 
Lösungselementen im Mittelpunkt. 

p.289
Lernförderliche Software 

Aus den Interviews mit den begleitenden Lehrern und den Unterrichtsbeobach-
tungen konnten konkrete Anforderungen für die lernförderliche Software abge-
leitet werden. Zum Verstehen von Internetanwendungen und -diensten ist es 
notwendig, sowohl sichtbare als auch nicht sichtbare informatische Konzepte 
und Komponenten für Schüler transparent zu machen. Für einen handlungs-
orientierten Zugang eignen sich interaktive Unterrichtsmittel, die eine Verbin-
dung  von  mehreren  Perspektiven  auf  den  Lerngegenstand  ermöglichen.  Ste-
chert und Schubert (2007) lieferten einen Ansatz für das Verstehen von Infor-
matiksystemen, der insbesondere die Verknüpfung von nach außen sichtbarem 
Verhalten eines Informatiksystems mit der inneren Struktur fördert.  

Auf  der  Grundlage  der  Erkenntnisse  aus  den  Unterrichtsprojekten  wurde  an 
der Universität Siegen eine Lernsoftware mit dem Schwerpunkt Strukturen des 
Internets entwickelt (Asschoff et al., 2007).  

Die Lernsoftware (URL: http://www.die.informatik.uni-siegen.de/pgfilius) un-
terstützt den Sichtenwechsel (Abb. 12.6). Mit der Netzwerkansicht[...] 
Die Anwendungssicht [...]. Eine Nach-
richtensicht zeigt dazu den Datenaustausch zwischen Rechnern und Programmen 
an.  Mit  einer  Quelltextsicht  ist  es  außerdem  möglich,  eigene  Anwendungen  zu 
konstruieren. Das nach außen sichtbare Verhalten der Internetanwendungen wird 
somit durch die Anwendungssicht dargestellt. Netzwerksicht und Nachrichten-
sicht gewähren Einblick in die innere Struktur des Informatiksystems. 

\textbf{Kompendium multimediales Lernen}

p.118
Adressatenanalyse: unter anderem: 
Vorwissen und relevante Erfahrungen: Was kann an theoretischem Wissen (Hinter-
grundwissen,  einschlägiges  theoretisches  Fachwissen),  an  Handlungswissen  und 
praktischen Erfahrungen vorausgesetzt werden? Je passgenauer die neuen Informati-
onen ausgewählt werden, umso unwahrscheinlicher werden Langeweile – weil vieles 
bereits bekannt ist – und Überforderung aufgrund von Wissenslücken. 

Wissens- und Aufgabenanalyse: unter anderem: 
Welche Fähigkeiten und welches Wissen sind notwendig, um den festgestellten Be-
darf zu befriedigen? Welche Inhalte sollen vermittelt werden? 

p.185
Orientierungsmarken: 
Mayer et al. (2005b) konnten in ihren Studien nachweisen, dass Lernende ein umfas-
senderes und detaillierteres Verständnis vom Lerngegenstand erweben, wenn in diesem 
Wesentliches hervorgehoben wurde (Signaling Principle).

p.270
Simulation

Eine Animation wird dann zur Simulation, wenn sie einen bestimmten Grad an Interak-
tivität aufweist. 

Rieber (2005) dagegen bestimmt den Begriff „Simulation“ nicht in Relation zur Ani-
mation. Er definiert Simulationen als Computerprogramme, die Phänomene oder Aktivi-
täten modellieren und die dafür vorgesehen sind, dass die Nutzer durch Interaktionen 
mit ihnen etwas über diese Phänomene und Aktivitäten lernen.

Im Gegensatz dazu 
bieten  Simulationen  die  Möglichkeit  zur  Interaktion  mit  dem  Lehrstoff  (learning  by 
doing). Der Schwerpunkt liegt hier auf der Exploration anstelle von Erläuterungen. 
Durch die Manipulation von dynamischen Elementen kann der Lernende die Konse-
quenzen erfahren und beobachten (Rieber, 2005). Er kann direkt beobachten, wie sich 
die von ihm vorgenommenen Veränderungen auswirken 

Simulationen  haben  von  sich  aus  kein  bestimmtes  Lernziel  außer  der  Exploration. 

Simulationen  können  Prozesse,  die  in  der  Natur  nicht  beobachtbar  sind,  nicht  nur 
sichtbar, sondern auch erfahrbar machen

Modellanwendende Simulationen 

Bei  modellanwendenden  Simulationen  wurde  das  zugrunde  liegende  mathematische 
Modell bereits programmiert und der Lernende kann einige Parameter verändern. 

Gestaltung von Erläuterungen 

Prinzip der zeitlichen Kontiguität :
Sie sollten zeitgleich mit der entsprechenden Animation präsentiert werden, damit die 
Inhalte gleichzeitig im Arbeitsgedächtnis sind und verarbeitet werden können.