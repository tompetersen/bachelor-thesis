\chapter{Fazit}

%Fazit 
%TLS eignet sich für die Hochschullehre
%TLS bietet genug Ansätze für Grundlagen/Fundamentale Ideen im Bereich der IT-Sicherheit
%Die Anwendung aufbauend auf Empfehlung der Fachdidaktik/E-Learning ist hilfreich, um die Abläufe in TLS besser zu verstehen.

In dieser Arbeit wurde gezeigt, dass das TLS-Protokoll sich für den Einsatz in der Hochschullehre im Bereich der IT-Sicherheit eignet. Hierfür spricht insbesondere die Möglichkeit, viele Grundlagen der IT-Sicherheit und Kryptographie am Beispiel von TLS erklären zu können, von denen einige in dieser Arbeit betrachtet wurden. Voraussichtlich werden diese Grundlagen in vielen Anwendungen und Bereichen auch in den nächsten Jahren ihre Bedeutung nicht verlieren.

Die auf Empfehlungen der Fachdidaktik zu explorativem Lernen und Simulationen basierend entwickelte Anwendung unterstützt das Verständnis der Funktionsweise von TLS. \todo{Anmerkung Papa}

%Weiterführende Arbeiten
%Erweiterungen der TLS-Simulation: Verkürzter Handshake, Zertifikatshandling, Daten selber 
%Erweiterung um weitere Protokolle -> einfach umsetzbar durch bestehenden Rahmen, der sich insbesondere um die graphische Aufbereitung und Interaktivität bereits kümmert.

Weiterführende Arbeiten könnten diese Anwendung auf zweierlei Arten erweitern. Erstens wäre die Implementierung einiger nicht umgesetzter Funktionen, die in Abschnitt \ref{sec_analysis_tls_plugin} beschrieben wurden, sinnvoll. Gerade die Implementierung des verkürzten Handshakes, der Validierung von Zertifikaten und das Versenden unterschiedlicher Daten würden sich hier anbieten.

Zweitens bietet der modulare und erweiterbar gehaltene Aufbau der Anwendung die Möglichkeit, auch andere Protokolle wie IPSec zu implementieren.\\
Aber auch das Verständnis anderer Verfahren wie beispielsweise Challenge-Response-Au"-then"-ti"-fi"-zie"-rung könnte durch die Anwendung erleichtert werden. Als Hilfestellung wird in Anhang \ref{cha_tutorial_plugin} dieser Arbeit anhand eines einfachen Beispiels auf Dinge hingewiesen, die bei der Erweiterung der Anwendung beachtet werden müssen.