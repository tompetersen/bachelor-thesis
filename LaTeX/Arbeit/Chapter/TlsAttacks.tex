
\chapter{Angriffe gegen SSL und TLS}

\label{cha_attacks}

Eine gute Übersicht zu bisherigen Angriffen auf TLS findet sich in \cite{meyer13}. Viele Schwächen früherer Protokollversionen bis SSL 3 sind in \cite{wagner96} zu finden.\todo{JEDEN Angriff auf angreifbare Version und Änderungen in neuen Versionen überprüfen.}

\section{Version Rollback}

\label{sec_version_rollback}

Ein Angreifer kann eine SSL 3.0-konforme \clienthello{}-Nachricht so modifizieren, dass der Server eine SSL 2.0-Verbindung aufbaut. So kann der Angreifer alle Schwächen der älteren Protokollversion ausnutzen. \todo{Durch Ändern des Versionsfeldes in der Nachricht?} In der SSL 3.0-Spezifikation (\cite{ssl30}) wurde vorgeschrieben, dass bestimmte Bytes des PKCS\#1-Paddings (siehe \cite{pkcs1}) einen festen Wert erhalten sollten, falls der Client SSL 3.0 unterstützt. So bleibt Kompatibilität mit der älteren Version erhalten, aber Version-Rollback-Angriffe werden trotzdem erkannt.\\
Ein Schwachpunkt könnte laut \cite{wagner96} immer noch die Wiederaufnahme einer SSL 3.0-Sitzung durch eine SSL 2.0 \clienthello{}-Nachricht sein. Dieses sollte in Implementierungen verhindert werden.

Auch in neueren Browsern kann dieser Angriff noch zum Problem werden, falls Fallback-Lösungen auf ältere Versionen implementiert sind, wenn ein Verbindungsversuch scheitert.

Durch die \finished{}-Nachricht ab SSL 3.0, die alle Handshake-Nachrichten authentifiziert, wird dieser Angriff verhindert. 

\section{Ciphersuite Rollback}

Ein gegen SSL 2.0 bestehender Angriff ermöglichte aktiven Angreifern die während des Handshake-Protokolls übertragenen Listen von unterstützten Cipher Suites zu verändern, so dass schwache kryptographische Verfahren erzwungen werden konnten (oftmals exportgeschwächte Verfahren mit kürzeren Schlüsselängen).

Ab SSL 3.0 wird dieser Angriff dadurch verhindert, dass die \finished{}-Nachrichten von Client und Server jeweils einen mit dem \mastersecret{} berechneten MAC über die Nachrichten des \handshakeprotocol{}s enthalten, der die Integrität dieser Nachrichten bestätigt.

Eine detaillierte Übersicht ist in \cite{wagner96} zu finden.

\section{Verhindern der \changecipherspec{}-Nachricht}

Für den Sonderfall einer SSL-Verbindung, die lediglich die Integrität der Nachrichten schützen soll, aber nicht verschlüsselt, lässt sich ausnutzen, dass der in der \finished{}-Nachricht gesendete MAC die \changecipherspec{}-Nachricht nicht mit einschließt. Dadurch kann ein aktiver Angreifer diese Nachrichten abfangen und nicht weiterleiten, sodass die Verbindungspartner die Integritätsprüfung nicht einsetzen. Ein Angreifer ist so in der Lage, gesendete Nachrichten zu verändern. 

Theoretisch wäre der Angriff unter bestimmten Voraussetzungen und schwacher Kryptographie auch bei verschlüsselten Verbindungen möglich. Dazu müsste die empfangene \finished{}-Nachricht vor dem Weiterleiten entschlüsselt werden. In bestimmten Fällen wäre zwar genug bekannter Klartext vorhanden, um einen Brute-Force-Angriff auf den Schlüssel zu erlauben, aber selbst bei exportgeschwächten 40-Bit Schlüsseln ist der Rechenaufwand hierfür sehr groß und unter praktischen Gesichtspunkten kaum unbemerkt ausführbar.

Alle TLS-Versionen verhindern diesen Angriff dadurch, dass sie eine \changecipherspec{}-Nachricht vor der \finished{}-Nachricht explizit vorschreiben.

Details zu diesem Angriff sind in \cite{wagner96} zu finden.

\section{Bleichenbacher-Angriff}
\label{sec_attack_bleichenbacher}

Daniel Bleichenbacher stellte 1998 in \cite{bleichenbacher98} einen Adaptive-Chosen-Ciphertext-Angriff gegen RSA-basierte Protokolle vor. Im Fall von SSL wird versucht, das \premastersecret{}, das während des Handshakes RSA-verschlüsselt gesendet werden kann, zu erhalten.

Der Angriff basiert auf dem festen Format nach PKCS\#1 (siehe \cite{pkcs1}) formatierter Nachrichten, wie in Abbildung \ref{fig_pcks_padding} dargestellt. Die ersten beiden Bytes haben immer den gleichen Wert, danach folgen das Padding (bestehend aus zufälligen Bytes ungleich null) und die Daten, getrennt durch ein Nullbyte. Nachrichten in diesem Format werden als Integer interpretiert, per RSA verschlüsselt und versendet. Der Empfänger entschlüsselt die Nachricht, überprüft das Format des als Bytekette interpretierten Ergebnisses und kann dann die Daten wieder extrahieren.

\begin{figure}[H]
	\centering
	\begin{tikzpicture}[node distance=0cm, outer sep = 0pt]
		\tikzstyle{field}=[draw, rectangle, minimum height=1cm, minimum width=4cm, fill=blue!15, anchor=south west]
		\tikzstyle{byte}=[draw, rectangle, minimum height=1cm, minimum width=1.5cm, fill=gray!20, anchor=south west]

		\node[byte] (first) at (0,0) {0x00};
		\node[byte] (second) [right = of first] {0x02};
		\node[field] (pad) [right = of second] {Padding};
		\node[byte] (sep) [right = of pad] {0x00};
		\node[field] (data) [right = of sep] {Daten};
	\end{tikzpicture}
	\caption{PCKS \#1-Format}
	\label{fig_pcks_padding}
\end{figure}

Voraussetzung für diesen Angriff ist der Zugriff auf ein Orakel, das dem Angreifer für eine verschlüsselte Nachricht lediglich mitteilt, ob das Padding der entschlüsselten Nachricht korrektes Format besitzt.

Im Folgenden sei \((n,e)\) ein öffentlicher RSA-Schlüssel und \((n,d)\) der zugehörige geheime Schlüssel. Der Angreifer möchte eine Nachricht \(m \equiv c^d\mod{n}\) erhalten, für die er im Besitz von \(c\) ist.\\ 
Dazu wählt er eine Zahl \(s\), berechnet \(c' \equiv cs^e \mod{n}\) und sendet \(c'\) an das Orakel. Wenn das Orakel korrektes Format signalisiert, dann weiß der Angreifer, dass die ersten zwei Bytes von \[(c')^d \equiv (cs^e)^d \equiv c^ds \equiv ms\mod{n}\] \(0x00\) und \(0x02\) sind. Mit diesem Wissen lässt sich ein neuer Wert \(s\) wählen, der weitere Informationen über \(m\) enthüllt. Details zu diesem iterativen Verfahren sind in \cite{bleichenbacher98} zu finden. Der Autor schätzt die Anzahl an nötigen Orakelanfragen mit etwa \(2^{20}\) ab.\\
Das Orakel lässt sich auf zwei Weisen erhalten. Entweder gibt die Implementierung detaillierte Fehlermeldungen über ungültiges PKCS-Format zurück oder ermöglicht durch Zeitunterschiede bei der Verarbeitung gültiger und ungültiger Nachrichten einen Timing-Angriff. 

In TLS ab Version 1.0 wird der Angriff dadurch verhindert, dass bei ungültigem PKCS-Format ein zufälliges \premastersecret{} erzeugt wird, mit dem der Handshake fortgesetzt wird. Dadurch scheitert der Handshake erst bei der Überprüfung der \finished{}-Nachricht und enthüllt keine Informationen über gültiges oder ungültiges Format.

\section{Padding-Orakel-Angriff}
\label{sec_attack_padding_oracle}

In \cite{vaudenay02} beschreibt der Autor einen Angriff zur Erlangung des Klartextes, bei dem das für Blockchiffren nötige Padding im CBC-Modus ausgenutzt wird. Durch das vorgegebene Format des Paddings und da das Padding bei TLS nicht durch den MAC geschützt ist (MAC - then PAD - then Encrypt) ermöglicht es theoretisch in einer relativ kleinen Zahl von Anfragen die Berechnung des Klartextes.\\
Praktisch konnte das Verfahren nicht eingesetzt werden, da SSL 3.0 für Paddingfehler und MAC-Überprüfung gleiche Fehlermeldungen (\badrecordmac{}) ausgibt. In TLS 1.0 und 1.1 gibt es getrennte Fehlermeldungen (\badrecordmac{} und \decryptionfailed{}), so dass der Angriff theoretisch möglich wäre. Allerdings werden die Fehler über das \alertprotocol{} verschlüsselt gesendet, so dass ein Angreifer die Fehlerart anders (beispielsweise über Log-Einträge) erhalten muss. TLS 1.2 verbietet das Senden von \decryptionfailed{}-Fehlern aus diesem Grund.\\
Ein weiterer Nachteil ist, dass es sich bei \badrecordmac{}- und \decryptionfailed{}-Fehlern um fatal alerts handelt, die zum Abbruch der Sitzung führen.
%Nichtsdestotrotz bietet die hier beschriebene Methode eine Grundlage für weitere Angriffe.

Das verwendete Padding besteht immer aus Bytes mit dem Wert \(n\), wobei \(n\) die nötige Anzahl an Paddingbytes bis zum Erreichen eines Vielfachens der Blocklänge bezeichnet. Das Padding kann also folgende Werte annehmen: \(1, 22, 333, 4444, \dots\). Voraussetzung für diesen Angriff ist der Zugriff auf ein Orakel, das dem Angreifer für eine verschlüsselte Nachricht lediglich mitteilt, ob das Padding der entschlüsselten Nachricht korrektes Format besitzt. Im Folgenden bezeichnet \(C(x)\) die Verschlüsselung des Blockes \(x\) und \(C^{-1}(y)\) die Entschlüsselung von \(y\).\\
Wenn der Angreifer nun das letzte Byte eines Chiffretextblocks \(y\) erhalten möchte, so sendet er \(r | y\) \todo{+ als Konkatenation?} mit \(r = r_1,  \dots , r_b\) als zufällige Bytes und \(b\) als Blocklänge (in Byte) an das Orakel. Bei der Entschlüsselung im CBC-Modus wird der letzte Chiffretextblock (hier \(y\)) entschlüsselt und mit dem vorletzten Block XOR-verknüpft, um den Klartextblock zu erhalten. Dieser Block (hier \(x\)) wird dann auf gültiges Padding überprüft:
\[x=C^{-1}(y) \oplus r\]
Wenn das Orakel gültiges Padding signalisiert, dann ist am wahrscheinlichsten, dass \(x\) auf \(1\) endet und somit das letzte Byte von \(C^{-1}(y)= r_b \oplus 1\) ist. Bei ungültigem Padding wird ein neuer Wert \(r_b\) gewählt und das Orakel neu befragt.\\
In \cite{vaudenay02} wird ein Algorithmus, mit dem auch die unwahrscheinlicheren Fälle von längerem Padding abgedeckt werden, und ein Verfahren, um aus dem letzten Byte einen kompletten Block zu erhalten, angegeben. 

In \cite{canvel03} beschreiben die Autoren eine Umsetzung des Angriffs auf TLS-gesicherte IMAP-Verbindungen zur Erlangung von Passwörtern. Hierbei wird das Problem ununterscheidbarer und verschlüsselter Fehlermeldungen durch einen Timing-Angriff umgangen. Außerdem bedenken die Autoren das Abbrechen der Sitzung durch Nutzung vieler paralleler Sitzungen mit dem gleichen verschlüsselten Aufruf (wie es bei der Authentifizierung im IMAP-Protokoll der Fall ist).

%Nicht-Überprüfen des Paddings auch keine Option, siehe https://www.openssl.org/~bodo/tls-cbc.txt (Letzte EMail)

\section{Lucky Thirteen}

In \cite{paterson13} stellen die Autoren weitere auf \cite{vaudenay02} basierende Angriffe vor, die ebenfalls auf Timing-Angriffen zur Erkennung falschen Paddings und mehrere Verbindungen setzen.

\section{Chosen-Plaintext-Angriff gegen bekannte IVs}

\label{sec_known_ivs}

In \cite{bard04} stellt der Autor einen Angriff vor, der die Art ausnutzt, wie die für den CBC-Modus nötigen Initialisierungsvektoren (IV) von TLS bereitgestellt wurden. Durch die Nutzung des letzten Ciphertextblocks der letzten Nachricht als IV der neuen Nachricht lässt sich unter bestimmten Voraussetzungen ein Chosen-Plaintext-Angriff durchführen. Der Autor beschreibt eine Möglichkeit unter Nutzung von Browser-Plugins über HTTPS übertragene Passwörter oder PINs herauszufinden. In \cite{bard06} verbessert der Autor seinen Angriff durch die Nutzung von Java-Applets anstelle von Browser-Plugins.

Der eigentliche Angriff entspricht dem folgenden Prinzip: Wenn eine Nachricht \(C = C_0,\dots,C_l\) gesendet wurde, wird für die nächste Nachricht \(C_l\) als IV verwendet werden. Wenn der Angreifer überprüfen möchte, ob ein Klartextblock \(P^*=P_j\) zu \(C_j\) verschlüsselt wurde, so bringt er einen Sender dazu eine Nachricht \(P'\) mit dem ersten Block \(P_1'=C_{j-1} \oplus C_l \oplus P^*\) zu verschlüsseln und erhält als ersten Chiffretextblock:
\begin{align*}
C_1' &= C_{K}(P_1' \oplus \text{IV})\\
	&= C_{K}(P_1' \oplus C_l)\\
	&= C_{K}(C_{j-1} \oplus C_l \oplus P^* \oplus C_l)\\
	&= C_{K}(C_{j-1} \oplus P^*)
\end{align*}
Außerdem gilt auch \(C_j= C_{K}(P_j \oplus C_{j-1})\). Der Angreifer kann nun überprüfen, ob \(C_1'=C_j\) und damit \(P^*=P_j\) gilt, ob also seine Wahl für den gesuchten Klartextblock stimmt.

Seit TLS 1.1 werden explizite IV vorgeschrieben. Hierzu besteht jede verschlüsselte Nachricht aus einem Block mehr als Klartextblöcken. Dieser erste Block bildet den IV für die restliche Verschlüsselung. Da dieser IV nicht vor dem Empfang der Nachricht bekannt ist, wird der hier beschriebene Chosen-Plaintext-Angriff verhindert.

\section{BEAST}

In \cite{duong11} und in einem Konferenzbeitrag auf der ekoparty Security Conference 2011 wurde von den Autoren das Tool BEAST vorgestellt, das die Ideen aus \cite{bard04} aufgreift. Die Autoren erweiterten den Angriff jedoch auf einen sogenannten block-wise chosen-boundary Angriff, bei dem der Angreifer die Lage der Nachricht in den verschlüsselten Blöcken verändern kann. Die Autoren zeigten auch die praktische Umsetzbarkeit am Beispiel des Entschlüsselns einer über HTTPS gesendeten Session-ID.
% http://www.ekoparty.org/archivo.php
% http://vnhacker.blogspot.co.uk/2011/09/beast.html

%http://blog.ivanristic.com/2013/09/is-beast-still-a-threat.html %??

\section{CRIME}

\label{sec_attack_crime}

Auf der ekoparty Security Conference 2012 stellten die Entdecker des BEAST-Angriff einen weiteren Angriff vor, der die (optionale) Kompression in TLS nutzt, um beispielsweise Cookiedaten zu stehlen. 
%Kein Paper, Folien hier: https://docs.google.com/presentation/d/11eBmGiHbYcHR9gL5nDyZChu_-lCa2GizeuOfaLU2HOU/edit#slide=id.g1d134dff_1_162
%http://security.stackexchange.com/questions/19911/crime-how-to-beat-the-beast-successor

Dabei wird ausgenutzt, dass Kompressionsalgorithmen bereits verwendete Zeichenketten beim erneuten Auftreten verkürzen. Wird nun beispielsweise ein HTTP-Header wie \monospace{Cookie: twid= secret} gesendet, kann ein Angreifer durch das Einbringen von \monospace{Cookie: twid= a...} und \monospace{Cookie: twid= s...} einen Längenunterschied der Nachrichten feststellen und so das Geheimnis zeichenweise erhalten.

Als Folge wurde Kompressionsunterstützung in Firefox und Chrome deaktiviert. Kompression ist im aktuellen Draft von TLS 1.3 (\cite{tls13}) nicht mehr enthalten.

\section{Poodle}

In \cite{moeller14} nutzen die Autoren den erneuten Verbindungsversuch mit älteren Protokollversionen, wenn der Handshake fehlschlägt (SSL 3.0-Fallback), der in vielen TLS-Implementierungen eingesetzt wird. Darauf aufbauend beschreiben sie einen Angriff, der bestehende Schwächen in der RC4-Chiffre \todo{RC4 noch irgendwo unterbringen? Vllt bei den Ciphersuites? http://www.isg.rhul.ac.uk/tls/} bzw. in der Nicht-Prüfung von Padding im CBC-Modus in SSL 3.0 ausnutzt, um Cookiedaten zu stehlen.
%https://poodlebleed.com/
Von der Unterstützung von SSL 3.0 wird in \cite{deprecate_ssl30} abgeraten.

\section{FREAK}

\label{sec_attack_freak}

Eine Gruppe von Pariser Wissenschaftlern entdeckte eine Möglichkeit, wie ein Angreifer die Kommunikationspartner während des Handshakes zur Nutzung schwacher Kryptographie (RSA export cipher suite) bringen kann. Weiterhin zeigten sie in \cite{freak15} die Machbarkeit der Faktorisierung der entsprechenden RSA-Module und die Praxistauglichkeit des Angriffs.
%https://freakattack.com/
%https://www.smacktls.com/#freak
%http://blog.cryptographyengineering.com/2015/03/attack-of-week-freak-or-factoring-nsa.html

Der Angriff beruht auf Fehlern in TLS-Implementierungen, die schwache RSA-Schlüssel vom Server akzeptieren, selbst wenn sie die \ciphersuites{} nicht anbieten.

\section{logjam}

In \cite{logjam15} beschreiben die Autoren mehrere Angriffe gegen die Nutzung von Diffie-Hellman-Schlüsselaustausch während des TLS Handshakes. Ein Angriff richtet sich gegen kleine DH-Parameter (DHE-EXPORT), ein weiterer nutzt die weite Verbreitung von standardisierten DH-Parametern, um mittels Vorberechnung bestimmter Werte schneller diskrete Logarithmen für beim DH-Verfahren gesendete Nachrichten zu berechnen.
%weakdh.org

\section{Zertifikate und Verwandtes}

\label{sec_certificates}

Viele Probleme, die in den letzten Jahren aufgetreten sind, betreffen nicht das TLS-Protokoll direkt, sondern die Erstellung und Validierung von (insbesondere) Server-Zertifikaten, und seien deshalb nur am Rande erwähnt. Ein guter Überblick ist in \cite{meyer13} zu finden.

Oftmals richteten sich diese Angriffe gegen mangelnde Zertifikatvalidierung in TLS-Implementierungen (keine Validierung, keine Überprüfung des Servernamens, Akzeptanz unsignierter oder abgelaufener Zertifikate, ...) oder wenig Sorgfalt bei der Zertifikaterstellung durch Certificate Authorities (Nutzung von MD5, fehlerhafte Validierung von übermittelten Servernamen, fehlerhafte Ausgabe von intermediate-Zertifikaten, mangelhaft abgesicherte Server, ...).

Der Vollständigkeit halber sei hier auch noch die notwendige Sicherheit des privaten Serverschlüssels erwähnt. Gelangt ein Angreifer in seinen Besitz, so kann er den Datenverkehr problemlos mitlesen oder verändern.