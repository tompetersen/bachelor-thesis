\chapter{SSL und TLS - ein Überblick}

\section{Geschichte von TLS}

Dieser Überblick basiert in großen Teilen auf \cite{schmeh09}.

SSL (Secure Socket Layer) bzw. TLS\footnote{Im weiteren Verlauf dieser Arbeit wird der Einfachheit halber lediglich von TLS gesprochen. Bei etwaigen Unterschieden wird explizit auf diese eingegangen.} (Transport Layer Security) ist ein zustandsbehaftetes Protokoll, das auf dem TCP-Protokoll\footnote{DTLS (Datagram Transport Layer Security) ist ein auf TLS basierendes Protokoll, dass auf UDP aufsetzt.} der Transportschicht des TCP/IP-Protokollstapels aufbaut. 

Hauptaufgaben von TLS sind Authentifikation der Kommunikationspartner, Verschlüsselung der Kommunikation sowie die Sicherstellung der Integrität der übertragenen Nachrichten. Die hierbei verwendeten kryptographischen Verfahren werden erst zu Beginn der Kommunikation festgelegt. 

Viele Protokolle der Anwendungsschicht nutzen TLS zur sicheren Datenübertragung, so beispielsweise HTTPS oder FTPS und auch viele Anwendungen übertragen ihre Daten TLS-gesichert.

SSL wurde von der Firma Netscape entwickelt und nachdem es starke Verbreitung gefunden hatte, durch die IETF als TLS 1.0 standardisiert (TLS 1.0 entspricht SSL 3.1). Aktuell ist die TLS-Version 1.2 und an Version 1.3 wird gearbeitet.

\ifoptionfinal{}{
\begin{description}
	\item[TLS 1.0] RFC 2246 - http://tools.ietf.org/html/rfc2246
	\item[TLS 1.1] RFC 4346 - http://tools.ietf.org/html/rfc4346
	\item[TLS 1.2] RFC 5246 - http://tools.ietf.org/html/rfc5246
	\item[TLS 1.3] Draft - https://tools.ietf.org/html/draft-ietf-tls-tls13-05
	\item[TLS Extensions] Z.B.\\
		RFC 3546 - http://tools.ietf.org/html/rfc3546, \\
		RFC 3466 - http://tools.ietf.org/html/rfc4366, \\
		RFC 6066 - http://tools.ietf.org/html/rfc6066
\end{description}
}

\section{Implementierungen}

